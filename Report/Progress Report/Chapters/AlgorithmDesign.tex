\chapter{Algorithm Design}

This is some sample text.

\section{Scoring Rule}

In order to rank data according to its relevance, it must first be scored according to its relevance. Scoring may be considered a form of classification, in which an item of data is classified as belonging to a particular cluster within a data set, represented as a set of data points in a feature vector. Scoring may be done using supervised machine-learning approaches or using an unsupervised scoring function. 

The word 'importance' is used to describe the non-user-specific likely relevance of an item of data. 'Relevance' signifies a user-specific importance of an item of data.

\subsection{Topic-classification scoring formulae}

Scoring functions typically combine key-term statistics into a single score as a measure of the similarity between a query and a document.
Term frequency is the most frequently used and widely explored approach to understanding the relevance of a text. Term rank scoring formulae are comprised of a base formula (typically Okapi BM25, Eqn. \ref{OkapiBM25}) and a multiplicative or additive range variance term R \cite{OkapiBM25Paper}.

\begin{equation}\label{OkapiBM25}
	\sum\limits_{t\in d \bigcap q} \ln{\frac{N-df+0.5}{df+0.5}}\centerdot \frac{tf}{0.5+1.5 \centerdot \frac{dl}{avdl}+tf} \centerdot R
\end{equation}

Here $df$ is the document frequency (documents in a collection) and $tf$ is simply the term frequency (to occurrences of a term in a particular document). $avdl$ is the average document length and $R$ is a component which limits the range over which the term rank has an effect. 

Topic classification is done remotely using a similar technique using multiple term frequency using weights which describe how closely each term relates to the specified topic. This classification provides us with information about our item of data which can be used to rank it according to its relevance. 

\subsection{Unweighted Scoring Rule}

Having performed topic-classification on an item of data, a scoring algorithm must assign to it a value which is a measure of its relevance. Let us first consider the case of an unweighted rule which scores an item of data. 

A classified item of data (that is, one which has been assigned a score for each attribute according to how well that attribute describes the topic of the text), without considering the users preferences, has an important topic if the sum of the scores of each attribute is large. This is related to the average score of an attribute being large. Consequently, the importance of an item of data may be given as the average of the attribute scores

\begin{equation}\label{AverageUnweightedRule}	
	f(D) = \frac{\sum_{i=1}^{i} d_i}{n}
\end{equation}

where $D$ is our item of data, and the elements $d_i$ are its topic-attributes. 

\subsection{Weighted Scoring Rule}

Let us now move on to consider the user-specific case, or weighted case. Relevance is defined here as the extent to which the score of each attribute of an item of data, matches the preference score of the user for that attribute. Attribute-specific scores are found by comparing the complementary user and data-item attributes according to a given rule. A final scoring rule must be used to combine scores of individual attributes into an overall score for a particular item of data, to given its relevance.
Fagin and Wimmers \cite{FaginWimmers1} detail a formula for incorporating weights into scoring rule, no matter the nature of the scoring rule. It takes a function $f$ which describes an unweighted rule for applying a score to a tuple of $n$ entries (attributes in our case) and gives a weighted rule based on $f$ which is compatible with any rule. 

They take a set $\Theta = (\theta_1,\theta_2,\dotsc,\theta_n)$ of weights which relate the effect of an entry $x$ on the score of the tuple $ X = (x_1, x_2, \dots, x_n)$. If $f(X)$ is the unweighted scoring rule, then for a tuple of $m$ entries

\begin{equation}\label{WeightedRule}	
	f_\Theta (X) = \left(\sum_{i=1}^{m-1} i\centerdot (\theta_{\sigma{(i)}} - \theta_{\sigma{(i+1)}})\centerdot f(X\upharpoonright\left\{\sigma(1),\dots,\sigma(i)\right\})\right) + m \centerdot \theta_\sigma{(m)} \centerdot f(X)
\end{equation}

Here $X\upharpoonright\left\{\sigma(1),\dots,\sigma(i)\right\}$ is a restriction of $X$ to the domain of a bijection $\sigma$ which orders the weightings to match the order of entries in the tuple. In our case this bijection would be a mapping from each attribute of an item of data to the weighting associated with that attribute. The weighting $\Theta$ would be dependent upon a complementary tuple describing the user, i.e. their attribute preferences. The term $(\theta_{\sigma{(i)}} - \theta_{\sigma{(i+1)}})$ is the difference between the weightings of two consecutive entries. 

Fagin and Wimmers' equation allows me to include weighed relationships between attributes' sub-scores and the total score of the item of data. An item of data is represented as a tuple $ D = (d_1, d_2, \dots, d_n)$ where each entry $d_n$ is an attribute of the item of data. Given the context-sensitive nature of this project, attributes must affect the score of the tuple with differing degrees according to how well they match the preferences of the user. 

This idea of weighting will be an analogy for the users preferences, such that the greater the user's preference of an attribute, the greater the weighting will be. Thus the set of weights $\Theta$ becomes a tuple of entries $ U = (u_1, u_2, \dots, u_n)$ whereby $U$ represents the tuple of attributes depicting the user's preferences. This leads us the following result, showing a general solution for context-sensitive (weighted) scoring rule, for any unweighted rule $f(D)$.

\begin{equation}\label{OurWeightedRule}	
	f_U (D) = \left(\sum_{i=1}^{m-1} i\centerdot (u_{\sigma{(i)}} - u_{\sigma{(i+1)}})\centerdot f(D\upharpoonright\left\{\sigma(1),\dots,\sigma(i)\right\})\right) + m \centerdot u_\sigma{(m)} \centerdot f(D)
\end{equation}

Let us take the simple unweighted scoring rule whereby we compute the average of the the $i$ attributes $d_j$ in a tuple $D$ (Eqn. \ref{AverageUnweightedRule}). Using the weighting formula (Eqn. \ref{OurWeightedRule}) the weighted equivalent is

\begin{align}\label{OurWeightedRuleDerivation}
f_U (D) &= \left(\sum_{n=1}^{M-1} n\centerdot (u_{\sigma{(n)}} - u_{\sigma{(n+1)}})\centerdot \frac{\sum_{i=1}^{n} d_i}{n}\right) + M \centerdot u_\sigma{(M)} \centerdot \frac{\sum_{i=1}^{n} d_i}{n} 
\\ &= (u_1-u_2)d_1 + 2(u_2-u_3)\frac{d_1+d_2}{2} + 3u_3\frac{d_1+d_2+d_3}{3}
\\ &= u_1d_1 + u_2d_2 + \dots + u_Md_M
\end{align}

We are lead therefore to the proof of an intuitive linear weighted scoring rule (Eqn. \ref{BasicScoringRule}).

\begin{equation}\label{BasicScoringRule}
f_U (D) = \sum_{k=1}^{m-1} u_kd_k
\end{equation}

\subsection{Weighted Scoring Rule using Multiple Weights}

This project deals with the relevance of an item of data to a user. An item of data may be social media, news or productivity related. The relevance of productivity related information is not usually based upon its topic, for example the relevance of a calendar appointment for a meeting or a dentist appointment cannot be assessed according to the principles used above. Its topic-classification is not useful in commenting on its relevance. This situation calls for a weighting which is based not upon topic relevance, but upon time relevance. It must complement topic-relevance so as not to affect the ranking order of non-time-specific items of data. 

This weighting may be implemented using 
\begin{enumerate}
  \item a discrete binary decision (relevant at this point in time or not), or 
  \item a continuous function of increasing time-relevance as a deadline approaches
\end{enumerate}

The simplest of the two is the former, whereby an item of data (with a deadline in less than $t_threshold$ minutes) would be flagged as time-relevant and ranked at the top of the ordered list. 

The second implementation would be the preferable choice, since as the deadline approached, the item of data would creep up the ordered list until it reached the top at $t_threshold$ minutes before the deadline. 


\section{Lexical Analysis and Integration}

education/gender/readability etc.

\section{Ranking Algorithm}

i.e. Modified insertion sort