\title{A context-sensitive relevance-based intelligent data-ranking agent}
\author{ Thomas J. Bell}
\date{}
\documentclass[12pt]{article}
\usepackage[margin=1.in]{geometry}
\begin{document}
\maketitle
Supervisor: Mark Nixon

\paragraph{Problem}
Social media, productivity tools and internet-based information are abundant on mobile devices leading to users being overwhelmed with information, despite only a small amount of it having any interest to a particular individual at any given moment.

\paragraph{Goals}
The goal of this project is to provide a generalised ranking agent to order this vast range of information according to its context-specific relevance, given the user's personality, click-history and environment at any instance. This project will endeavour to abstract away a new ranking agent into a Java API for use in a variety of applications on a range of devices.

Existing ranking, sentiment analysis and data fusion algorithms will be investigated and adapted in order to produce a scalable and highly modular context-sensitive mobile-content relevance-based ranking agent.

 A personality profile and historical data will be used to maintain a user-context, which will behave like a search query. Topic analysis will be used to determine the data-context of a variety of available items of data, to be matched for relevance against the user-context. A stable context-sensitive ranking algorithm will be proposed and implemented to order data according to its context-specific relevance.

The Facebook and Twitter API's will be investigated to ensure cross-platform compatability and the Android, Spring MVC and Java Swing frameworks will be explored, to ensure compatability with Java mobile, web-based and desktop applications respectively. 

Realistic test data will be used throughout the development stages in a comprehensive variety of configurations. 
The primary deliverable will be a modular Java API. For the purpose of demonstration a mobile, web or desktop application will be designed and implemented, to showcase the ranking agent's capabilities. 

\paragraph{Scope}
For the purpose of allowing this project to focus on its main goals, it will not intend to improve upon existing sentiment/topic analysis algorithms at the outset, but rather use existing tools. 

Compatible items of data within the initial scope of this project include Facebook statuses/notifications, tweets, calendar appointments, tasks, emails and SMS messages and will allow for additional types of data to be added later.

The project will focus primarily on the ranking agent, designed to rank the items of data using modified algorithms on a specific set of feature vectors.

\end{document}